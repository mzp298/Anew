************************************************************************************

 A DETAILED DESCRIPTION OF ANALYTICAL EXPLOITATION ON UNIAXIAL CYCLE
 
\noindent              
************************************************************************************

\noindent
\textbf{Phase 1:} The deviatoric stress amplitude increases from $\sigma_y/s$ to $S_{max}$.

\noindent
The material is in local plastic regime, then $\dot{\varepsilon}^p>0$ and $\dot{\sigma}-\dot{b}=0$ $\Rightarrow$ $\dot{\Sigma}-\dfrac{E}{1+\nu}\dot{\varepsilon}^p=\dfrac{kE}{E-k}\dot{\varepsilon}^p$ $\Rightarrow$ 
$$\dot{\varepsilon}^p=\dfrac{(E- k)(1+\nu)}{E(E+k\nu)}\dot{\Sigma}.$$

\vspace{6pt}
\noindent
$\Rightarrow$ $\dot{\varepsilon}^p$ varies from 0 to $\dfrac{(E- k)(1+\nu)(S_{max}-\sigma_y/s)}{E(E+k\nu)}$.

\vspace{6pt}
\noindent
From Taylor-Lin scale transition model:
$$\dot{\sigma}=\dot{\Sigma}-\dfrac{E}{1+\nu}\dot{\varepsilon}_p=\dot{\Sigma}-\dfrac{E-k}{E-\nu k}\dot{\Sigma}=\dfrac{k(1-\nu)}{E-k\nu}\dot{\Sigma}.$$

\vspace{6pt}
\noindent
$\Rightarrow$ $\sigma$ varies from $\sigma_y/s$ to $\sigma_y/s+\dfrac{k(1-\nu)(S_{max}-\sigma_y/s)}{E-k\nu}$.

\vspace{6pt}
$$\dot{b}=\dot{\Sigma}-\dfrac{E}{1+\nu}\dot{\varepsilon}_p=\dot{\Sigma}-\dfrac{E-k}{E-\nu k}\dot{\Sigma}=\dfrac{k(1-\nu)}{E-k\nu}\dot{\Sigma}.$$

\vspace{6pt}
\noindent
$\Rightarrow$ $b$ varies from $0$ to $\dfrac{k(1-\nu)(S_{max}-\sigma_y/s)}{E-k\nu}$.

\vspace{6pt}
\noindent
So the energy dissipation rate is: $$(\sigma-b)\dot{\varepsilon}^p=\dfrac{\sigma_y}{s}\dot{\varepsilon}^p=\dfrac{\sigma_y}{s}\dfrac{(E- k)(1+\nu)}{E(E+k\nu)}\dot{\Sigma}.$$

\noindent
The energy dissipation is: $$(\sigma-b)\Delta\varepsilon^p=\dfrac{\sigma_y}{s}\dfrac{(E- k)(1+\nu)(S_{max}-\sigma_y/s)}{E(E+k\nu)}.$$

\vspace{6pt}
\noindent
\textbf{Phase 2:} The deviatoric stress amplitude decreases from $S_{max}$ to $S_{max}-2\sigma_y/s$.

\noindent
The material is in local elastic regime, then $\dot{\varepsilon}^p=0$ and $\dot{\sigma}-\dot{b}=0$ $\Rightarrow$

\vspace{6pt}
\noindent
$\dot{b}=0$, $\dot{\sigma}=\dot{\Sigma}-\dfrac{E}{1+\nu}\dot{\varepsilon}_p=\dot{\Sigma}$.

\vspace{6pt}
\noindent
$\sigma$ varies from $\sigma_y/s+\dfrac{k(1-\nu)(S_{max}-\sigma_y/s)}{E-k\nu}$ to $-\sigma_y/s+\dfrac{k(1-\nu)(S_{max}-\sigma_y/s)}{E-k\nu}$.

\vspace{6pt}
\noindent
$\sigma-b$ varies from $\sigma_y/s$ to $-\sigma_y/s$.

\vspace{6pt}
\noindent
The energy dissipation rate is: $$(\sigma-b)\dot{\varepsilon}^p=0.$$

\vspace{6pt}
\noindent
\textbf{Phase 3:} The deviatoric stress amplitude decreases from $S_{max}-2\sigma_y/s$ to $-S_{max}$.

\noindent
The material is in local plastic regime, then $\dot{\varepsilon}^p>0$ and $\dot{\sigma}-\dot{b}=0$ $\Rightarrow$ 
$$\dot{\varepsilon}^p=\dfrac{(E- k)(1+\nu)}{E(E+k\nu)}\dot{\Sigma}$$ as opposite to phase 1 for $\dot{\Sigma}<0$.

\vspace{6pt}
\noindent
$\Rightarrow$ $\varepsilon^p$ varies from $\dfrac{(E- k)(1+\nu)(S_{max}-\sigma_y/s)}{E(E+k\nu)}$ to 

\noindent
$\dfrac{(E- k)(1+\nu)(S_{max}-\sigma_y/s-S_{max}-(S_{max}-2\sigma_y/s))}{E(E+k\nu)}=-\dfrac{(E- k)(1+\nu)(S_{max}-\sigma_y/s)}{E(E+k\nu)}$.

\vspace{6pt}
\noindent
From Taylor-Lin scale transition model:
$$\dot{\sigma}=\dot{\Sigma}-\dfrac{E}{1+\nu}\dot{\varepsilon}_p=\dot{\Sigma}-\dfrac{E-k}{E-\nu k}\dot{\Sigma}=\dfrac{k(1-\nu)}{E-k\nu}\dot{\Sigma}.$$

\vspace{6pt}
\noindent
$\Rightarrow$ $\sigma$ varies from $-\sigma_y/s+\dfrac{k(1-\nu)(S_{max}-\sigma_y/s)}{E-k\nu}$ to $-\sigma_y/s-\dfrac{k(1-\nu)(S_{max}-\sigma_y/s)}{E-k\nu}$.

\vspace{6pt}
$$\dot{b}=\dot{\Sigma}-\dfrac{E}{1+\nu}\dot{\varepsilon}_p=\dot{\Sigma}-\dfrac{E-k}{E-\nu k}\dot{\Sigma}=\dfrac{k(1-\nu)}{E-k\nu}\dot{\Sigma}.$$
\vspace{6pt}
\noindent
$\Rightarrow$ $b$ varies from $\dfrac{k(1-\nu)(S_{max}-\sigma_y/s)}{E-k\nu}$ to $-\dfrac{k(1-\nu)(S_{max}-\sigma_y/s)}{E-k\nu}$.

\vspace{6pt}
\noindent
So the energy dissipation rate is: $$(\sigma-b)\dot{\varepsilon}^p=-\dfrac{\sigma_y}{s}\dot{\varepsilon}^p=-\dfrac{\sigma_y}{s}\dfrac{(E- k)(1+\nu)}{E(E+k\nu)}\dot{\Sigma}.$$

\noindent
The energy dissipation is: $$(\sigma-b)\Delta\varepsilon^p=-\dfrac{\sigma_y}{s}\dfrac{(E- k)(1+\nu)(-2S_{max}+2\sigma_y/s)}{E(E+k\nu)}=\dfrac{2\sigma_y}{s}\dfrac{(E- k)(1+\nu)(S_{max}-\sigma_y/s)}{E(E+k\nu)}.$$



\vspace{6pt}
\noindent
\textbf{Phase 4:} The deviatoric stress amplitude increases from $-S_{max}$ to $-S_{max}+2\sigma_y/s$.

\noindent
The material is in local elastic regime, then $\dot{\varepsilon}^p=0$ and $\dot{\sigma}-\dot{b}=0$ $\Rightarrow$

\vspace{6pt}
\noindent
$\dot{b}=0$, $\dot{\sigma}=\dot{\Sigma}-\dfrac{E}{1+\nu}\dot{\varepsilon}_p=\dot{\Sigma}$.

\vspace{6pt}
\noindent
$\sigma$ varies from $-\sigma_y/s-\dfrac{k(1-\nu)(S_{max}-\sigma_y/s)}{E-k\nu}$ to $\sigma_y/s-\dfrac{k(1-\nu)(S_{max}-\sigma_y/s)}{E-k\nu}$.

\vspace{6pt}
\noindent
$\sigma-b$ varies from $-\sigma_y/s$ to $\sigma_y/s$.

\vspace{6pt}
\noindent
So the energy dissipation rate is: $$(\sigma-b)\dot{\varepsilon}^p=0.$$


\vspace{6pt}
\noindent
\textbf{Phase 5:} The deviatoric stress amplitude increases from $-S_{max}+2\sigma_y/s$ to $\sigma_y/s$.

\noindent
The material is in local plastic regime, then $\dot{\varepsilon}^p>0$ and $\dot{\sigma}-\dot{b}=0$ $\Rightarrow$ 
$$\dot{\varepsilon}^p=\dfrac{(E- k)(1+\nu)}{E(E+k\nu)}\dot{\Sigma}$$ as in phase 1.

\vspace{6pt}
\noindent
$\Rightarrow$ $\dot{\varepsilon}^p$ varies from $-\dfrac{(E- k)(1+\nu)(S_{max}-\sigma_y/s)}{E(E+k\nu)}$ to $0$.

\vspace{6pt}
$$\dot{\sigma}=\dot{\Sigma}-\dfrac{E}{1+\nu}\dot{\varepsilon}_p=\dot{\Sigma}-\dfrac{E-k}{E-\nu k}\dot{\Sigma}=\dfrac{k(1-\nu)}{E-k\nu}\dot{\Sigma}.$$

\vspace{6pt}
\noindent
$\Rightarrow$ $\sigma$ varies from $\sigma_y/s-\dfrac{k(1-\nu)(S_{max}-\sigma_y/s)}{E-k\nu}$ to $\sigma_y/s$.

\vspace{6pt}
$$\dot{b}=\dot{\Sigma}-\dfrac{E}{1+\nu}\dot{\varepsilon}_p=\dot{\Sigma}-\dfrac{E-k}{E-\nu k}\dot{\Sigma}=\dfrac{k(1-\nu)}{E-k\nu}\dot{\Sigma}.$$
\vspace{6pt}
\noindent
$\Rightarrow$ $b$ varies from $-\dfrac{k(1-\nu)(S_{max}-\sigma_y/s)}{E-k\nu}$ to $0$.

\vspace{6pt}
\noindent
So the energy dissipation rate is: $$(\sigma-b)\dot{\varepsilon}^p=\dfrac{\sigma_y}{s}\dot{\varepsilon}^p=\dfrac{\sigma_y}{s}\dfrac{(E- k)(1+\nu)}{E(E+k\nu)}\dot{\Sigma}.$$

\noindent
The energy dissipation is: $$(\sigma-b)\Delta\varepsilon^p=\dfrac{\sigma_y}{s}\dfrac{(E- k)(1+\nu)(S_{max}-\sigma_y/s)}{E(E+k\nu)}.$$


From the three phase analysis in local plastic regime, the dissipated energy is like $dW(phase1)=\dfrac{1}{2}dW(phase3)=dW(phase5)$ and the dissipation rate is like $d\dot{W}(phase1)=d\dot{W}(phase3)=d\dot{W}(phase5)$.
\begin{equation}d\dot{W}=\dfrac{(E-k)(1+\nu) }{E(E-k\nu)}\left( \dfrac{\sigma_y}{s}\right) \left| \dot{\Sigma}\right|   
\end{equation}

